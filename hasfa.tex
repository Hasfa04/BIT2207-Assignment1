\documentclass{article}


\begin{document}

		\title{The Perfect Match}
		\author{Author : NANSAMBU HASIFA }
		\date{Student No: 215005121}
		\maketitle
	
\tableofcontents



\section{Introduction}
     “Life is like a box of chocolates. You never know what you are going to get." However, when you watch The Perfect Match starring Terrence Jenkins as Charlie and Cassie Ventura as Eva, you know you are going to get a lot of laughs and even a few tears. The screenplay for the 2016 movie was written by Gary Hardwick, Dana Verde and Brandon Broussard, tells the story of how this unrepentant man opens his heart to love. Some of it is so predictable you could set your watch by it, but there is a welcome and surprising layer of complexity running through the film that makes it a little bit more than your standard fare. The likeable and funny ensemble helps too.

\subsection{Body}
    
Charlie is a talent agent with a gift for taking care of his clients, diffusing their melt-downs, and seeing the potential in YouTube/Vine/Snapchat stars. Charlie has got a great number of friends, all of whom get their own story-lines in the film. There’s Victor (Robert Christopher Riley) and Ginger (Lauren London), together since they were kids and now planning on their wedding. Ginger makes more money than Victor does, and he’s not okay with it. Pressie (Dascha Polanco) and Rick (Donald Faison) are trying to have a baby. Charlie’s friends make his single status a big deal but he insists that he’s happy the way he is. His sister Sherry (Paula Patton) is a therapist who is convinced that unresolved grief at the death of their parents is why Charlie shut the door to his heart. However when his friends challenge him to date only one woman in the period leading to Victor and Ginger’s wedding, Charlie agrees and seconds later he has a juice-bar meet with Eva.
Eva has only been in long term relationships, but says she wants to try something fun and no-strings. They get intimate not only at his house but in restaurant bathrooms and it is all good, until he starts to develop feelings for her, mainly because she shows interest in photography and listens to him when he complains about his job. He never asks anything about herself. When things go sour, Charlie becomes an alcoholic and a mess which makes him get fired from his job as a talent agent.
There are awkward sequences where the comedy falls flat, and there are some sub-plots that don’t really add much. Bile Woodruff, a music video director who has helmed a couple of follow-up films in the popular franchises, keeps the story zipping along, though, and any time all of the friends are onscreen together, the film is successful because of the real dynamic they create. 



\section{Conclusion}
 “The Perfect Match” dominates so much of our lives that all it takes to be happy is to find “the right person”. Supposedly there’s “right person” for everyone out there once you stop needing a relationship and you start being okay with being alone, the right person will appear. But life is messier than that and some people don’t find their “perfect match”.



	
\end{document}